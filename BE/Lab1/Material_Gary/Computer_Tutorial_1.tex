%2multibyte Version: 5.50.0.2960 CodePage: 1252

\documentclass{article}
%%%%%%%%%%%%%%%%%%%%%%%%%%%%%%%%%%%%%%%%%%%%%%%%%%%%%%%%%%%%%%%%%%%%%%%%%%%%%%%%%%%%%%%%%%%%%%%%%%%%%%%%%%%%%%%%%%%%%%%%%%%%%%%%%%%%%%%%%%%%%%%%%%%%%%%%%%%%%%%%%%%%%%%%%%%%%%%%%%%%%%%%%%%%%%%%%%%%%%%%%%%%%%%%%%%%%%%%%%%%%%%%%%%%%%%%%%%%%%%%%%%%%%%%%%%%
\usepackage{amsmath}

\setcounter{MaxMatrixCols}{10}
%TCIDATA{TCIstyle=LaTeX article (bright).cst}

%TCIDATA{OutputFilter=LATEX.DLL}
%TCIDATA{Version=5.50.0.2960}
%TCIDATA{Codepage=1252}
%TCIDATA{<META NAME="SaveForMode" CONTENT="1">}
%TCIDATA{BibliographyScheme=Manual}
%TCIDATA{Created=Monday, September 28, 2009 18:38:22}
%TCIDATA{LastRevised=Friday, January 22, 2016 12:11:36}
%TCIDATA{<META NAME="GraphicsSave" CONTENT="32">}
%TCIDATA{<META NAME="DocumentShell" CONTENT="Exams and Syllabi\SW\Assignment">}
%TCIDATA{Language=American English}

\setlength{\topmargin}{-1.0in}
\setlength{\textheight}{9.25in}
\setlength{\oddsidemargin}{0.0in}
\setlength{\evensidemargin}{0.0in}
\setlength{\textwidth}{6.5in}
\def\labelenumi{\arabic{enumi}.}
\def\theenumi{\arabic{enumi}}
\def\labelenumii{(\alph{enumii})}
\def\theenumii{\alph{enumii}}
\def\p@enumii{\theenumi.}
\def\labelenumiii{\arabic{enumiii}.}
\def\theenumiii{\arabic{enumiii}}
\def\p@enumiii{(\theenumi)(\theenumii)}
\def\labelenumiv{\arabic{enumiv}.}
\def\theenumiv{\arabic{enumiv}}
\def\p@enumiv{\p@enumiii.\theenumiii}
\pagestyle{plain}
\setcounter{secnumdepth}{0}
\parindent=0pt
\input{tcilatex}
\begin{document}


\begin{center}
\bigskip

\textbf{Computer Tutorial 1: Matlab Basics}
\end{center}

\bigskip

Most Bayesians econometricians create their own programs and Matlab is
probably the most popular language for doing so. When writing my textbooks
and monograph I\ used Matlab and the websites associated with them contain
many sample programs. Matlab is a powerful program that is used in a variety
of scientific disciplines. In this course, I will not have time to teach you
everything about Matlab. Rather I\ attempt to teach you the basics necessary
to for you to carry out Bayesian inference in the models considered in this
course. With a knowledge of these basics, you can then use Matlab online
help or manuals to figure out more complicated things. My way of teaching
the basics of Matlab will be indirect:\ I\ will provide sample programs and
then get you to figure out what each step does. Then I\ will ask you to
create similar programs. If you do not like this style of teaching and
prefer a more systematic tutorial about Matlab, there are many web-based
resources you should feel free to use. Wikiuniversity has a free online
course: https://en.wikiversity.org/wiki/MATLAB\_essential. Matlab itself has
online tutorials which you may want to experiment with. Many lecturers have
put together tutorial materials that you may find useful (e.g.
http://www.tcd.ie/Economics/TEP/2010/TEP0110.pdf).

For questions 4 through 6 Matlab code is available on the course website.
For the first three exercises, I want you to program things up yourself.

\bigskip

{\large Overview}

Matlab is a matrix programming language. Unlike other programs you may have
used (e.g. Microfit, Eviews, Stata, etc.), Matlab is exclusively a
programming language. For instance, in these other languages you would just
click on a button to do OLS or type a simple command (e.g. "regress y x").
In Matlab you have to actually program up OLS. The OLS\ estimator is given
as:

\begin{equation*}
\widehat{\beta }=\left( X^{\prime }X\right) ^{-1}X^{\prime }y.
\end{equation*}

In Matlab, the format for this command is:

bhat = inv(x'*x)*x'y;

and you would actually have to type this line and then run the program
containing it.

A Matlab program is a series of commands such as this. These commands will
be listed in a file called an m-file (with extension ".m" so a program name
might be myprogram.m). Matlab has a text editor which allows you to create
an m-file (click on \textquotedblleft File/New\textquotedblright ). Once you
have created and saved such a file using the Matlab editor, you can then run
the program. You can do this by clicking on the "run\textquotedblright\ icon
which has a green triangle in it (located at the top of the text editor).
Matlab has many different windows, but two of them are the most important:
the Command window (where you can type commands and where the results of the
program are displayed)\ and the text editor (where you write and run your
code). I will demonstrate these general features of Matlab in the computer
lab. Below I\ will focus on the program writing.

Lines which begin with \% are ignored by Matlab. Programmers often place
comments in their programs in this way to explain key steps in their
programs. Note also that each Matlab command ends in a ";" (although there
are some exceptions to this such as for/end/if statements which do not
require the \textquotedblleft ;\textquotedblright\ although it is harmless
if you add it).

\bigskip

\textbf{Exercise 1:\ }\textit{Basic Matrix Commands}

\textit{A sample program}

\%This is a simple program which illustrates Matlab matrix commands

x = [1 2 3; 4 5 6; 7 8 9];

disp x;

disp(x);

y = [11 12 13; 14 15 16; 17 18 19];

disp y;

disp(y);

z=x+y;

disp z ;

disp(z);

w=x-y;

disp w;

disp(w);

u= x*y;

disp u;

disp(u);

a=[x, y];

disp a;

disp(a);

b=[x; y];

disp b;

disp(b);

c=x(:,2);

disp c;

disp(c);

d=y(:,1);

disp d;

disp(d);

e=x(2,3);

disp e

disp(e);

\bigskip

a)\ Create the preceding program and run it in Matlab and examine the output
Matlab produces. Describe what each line of this program does.

b)\ Add a line to this program which creates a new matrix, f, which is the
transpose of x.

c)\ Add a line to this program which creates a new matrix, g, which is the
identity matrix plus x.

d)\ Add a line to this program which creates a new matrix, ginv, which is
the inverse of g. What happens if you try to take the inverse of x?

\bigskip

\textbf{Exercise 2:\ }\textit{OLS\ Estimation Using Artificial Data}

\qquad Matlab has many scripts (i.e. built-in little programs) that you can
call automatically as part of your program. (Note:\ You can also create your
own scripts). Here we show you how to use Matlab's scripts for random number
generation from the Normal and Uniform distributions to create an artificial
data set from the regression model:\ $y_{i}=\alpha +\beta x_{i}+\varepsilon
_{i}$ for $i=1,..,100$. We set $\alpha =1$, $\beta =2$ and let the $x_{i}$
and $\varepsilon _{i}$ be random draws from the $U\left( 0,1\right) $ and $%
N\left( 0,1\right) $ distribution, respectively.

\textit{A sample program}

\%This is a program which artificially creates a data set and then does OLS
estimation using it

\%First part of this program artificially simulates data

n=100;

alpha=1;

beta=2;

e = randn(n,1);

x=rand(n,1);

y=alpha + x*beta + e;

\%following line adds intercept to x. explain why

x=[ones(n,1), x];

\%Following part of the program does OLS estimation

bhat = inv(x'*x)*x'*y;

disp 'The OLS estimate of beta is';

disp(bhat);

\%the OLS residuals

resids = y - x*bhat;

\%The OLS estimate of the error variance

s2 = resids'*resids/(n-2);

disp 'The OLS estimator of the error variance is';

disp(s2);

\bigskip

a)\ Create this program and run it in Matlab and examine the output in
Matlab. Describe what each line of this program does.

b) Extend this program to calculate the R$^{2}$ of this regression and print
out the result.

c)\ Extend this program to calculate the covariance matrix of the OLS
estimators (i.e. $var\left( \widehat{\beta }_{OLS}\right) =s^{2}\left(
X^{\prime }X\right) ^{-1}$) and print out the result.

\bigskip

\textbf{Exercise 3: }\textit{For Loops and If Statements}

\qquad When doing Monte Carlo integration or Gibbs sampling we repeatedly
take draws from the posterior distribution. Matlab does this kind of
repeated action using constructs called "for loops".

\textit{A sample program}

\%This is a program which illustrates for loops and if statements

\%first create a column vector to work with

x=[1;2;7;5;9;3;6;9;1;11;1];

\%the following command sums up the elements of a column vector

xsum=sum(x);

xsum1=0;

for i=1:11

xsum1=xsum1 + x(i,1);

end

disp xsum;

disp(xsum);

disp xsum1;

disp(xsum1);

\%now illustrate the if command

xsum2=0;

for i=1:11

if x(i,1)\TEXTsymbol{>}4

xsum2=xsum2 + x(i,1);

end

end

disp xsum2

disp(xsum2);

\bigskip

a)\ Create this program and run it in Matlab and examine the output.
Describe what each line of this program does. In particular, why are xsum1
and xsum the same as one another? What does the "if" statement do?\ What is
"xsum2"?

b)\ The sample program \textit{sums} various column vectors. Modify this
program to calculate \textit{averages} (i.e. means).

\bigskip

\textbf{Exercise 4:\ }\textit{Drawing from Standard Distributions}

Simulation-based inference using algorithms such as the Gibbs sampler
requires the researcher to be able to draw from standard distributions. In
this exercise we discuss how MATLAB can be used to obtain draws from a
variety of standard continuous distributions. Specifically, we obtain draws
from the Uniform, Normal, Student-t, Beta, Exponential and Chi-squared
distributions (see the Appendix of Bayesian Econometric Methods for
definitions of these distributions). Using the Matlab program for this
exercise (Ex4.m), obtain sets of 10, 100 and 100,000 draws from the Uniform,
standard Normal, Student-t(3) (denoted $t\left( 0,1,3\right) $ in the
notation of the Appendix to the book), Beta(3,2), Exponential with mean 5
and $\chi ^{2}\left( 3\right) $ distributions. For each sample size
calculate the mean and standard deviation and compare these quantities to
the known means and standard deviations from each distribution.

\bigskip

\textbf{Exercise 5:\ }\textit{Monte Carlo Integration}

If the posterior density $p\left( \theta |y\right) $ takes the a familiar
form (e.g. a Normal or Student-t or Gamma or other distribution for which
computer algorithms exist to take random draws) then we can obtain $R$
i.i.d. draws of the parameters, which we denote $\theta ^{\left( r\right) }$%
, $r=1,...,R$. Usually, quantities of interest to the researcher are
functions of the model parameters. Let us call such a function $g\left(
\theta \right) $. The researcher would then often be interested in
calculating:%
\begin{equation*}
E\left( g\left( \theta \right) |y\right) =\int g\left( \theta \right)
p\left( \theta |y\right) d\theta
\end{equation*}%
Monte Carlo integration allows us to calculate integrals of this form. The
weak law of large numbers implies that%
\begin{equation*}
E\left( g\left( \theta \right) |y\right) \simeq \frac{\sum_{r=1}^{R}g\left(
\theta ^{\left( r\right) }\right) }{R}
\end{equation*}%
This means that the posterior mean of $g\left( \theta \right) $ can be
calculated by drawing from the posterior and then averaging functions of the
posterior draws. Exercise:\ Suppose $p\left( \theta |y\right) \sim N(1,4)$
and the quantity of interest is $g\left( \theta \right) =\theta ^{2}$. Use
Monte Carlo integration to calculate $E\left( \theta ^{2}\right) $. Code for
this question is in Ex5.m. Note:\ in this case, you know the correct answer
is $E\left( \theta ^{2}\right) =5$ (since the definition of variance tells
you that $var\left( \theta \right) =E\left( \theta ^{2}\right) -\left[
E\left( \theta \right) \right] ^{2}$ and, in this exercise, $var\left(
\theta \right) =4$ and $E\left( \theta \right) =1$), so you would not need
to have done Monte Carlo integration. Optional exercise:\ modify Ex5.m to
calculate the posterior mean for a more complicated quantity of interest for
which analytical results are not so easily available (e.g. calculate $\Pr
\left( \theta ^{2}>2\right) $ or $E\left( \ln \left( \theta \right) \right) $
or some other choice for $g\left( \theta \right) $).\bigskip

\textbf{Exercise 6:\ }\textit{Gibbs Sampling from the Bivariate Normal }

The purpose of this question is to learn about the properties of the Gibbs
sampler in a very simple case.

Assume that you have a model which yields a bivariate Normal posterior,%
\begin{equation*}
\left( 
\begin{tabular}{c}
$\theta _{1}$ \\ 
$\theta _{2}$%
\end{tabular}%
\ \right) \sim N\left( \left[ 
\begin{tabular}{c}
$0$ \\ 
$0$%
\end{tabular}%
\ \right] ,\left[ 
\begin{tabular}{cc}
$1$ & $\rho $ \\ 
$\rho $ & $1$%
\end{tabular}%
\ \right] \right) ,
\end{equation*}%
where $|\rho |<1$ is the (known)\ posterior correlation between $\theta _{1}$
and $\theta _{2}$.

(a)Write a program which uses Monte Carlo integration to calculate the
posterior means and standard deviations of $\theta _{1}$ and $\theta _{2}$.

(b)\ Write a program which uses Gibbs sampling to calculate the posterior
means and standard deviations of $\theta _{1}$ and $\theta _{2}$.

(c)\ Set $\rho =0$ and compare the programs from parts a)\ and b) for a
given number of replications (e.g. $R=100$) and compare the accuracy of the
two algorithms.

(d)\ Repeat part (c)\ of this question for $\rho =.5,.9,.99$ and $.999$.
Discuss how the degree of correlation between $\theta _{1}$ and $\theta _{2}$
affects the performance of the Gibbs sampler. Make graphs of the Monte Carlo
and Gibbs sampler replications of $\theta _{1}$ (i.e. make a graph with
x-axis being replication number and y-axis being $\theta _{1}$). What can
the graphs you have made tell you about the properties of Monte Carlo and
Gibbs sampling algorithms?

(e) Repeat parts (c)\ and (d) using\ more replications (e.g. $R=50,000$)\
and discuss how Gibbs sampling accuracy improves with number of replications.

\end{document}
